% LaTeX file for resume 
% This file uses the resume document class (res.cls)

\documentclass[line,margin]{res} 
\usepackage{verbatim} 
\usepackage{url}
\usepackage{hyperref}
% the margin option causes section titles to appear to the left of body text 
\textwidth=5.5in % increase textwidth to get smaller right margin
%\usepackage{helvetica} % uses helvetica postscript font (download helvetica.sty)
%\usepackage{newcent}   % uses new century schoolbook postscript font 

\begin{document} 

\name{Ashesh Jain}
% \address used twice to have two lines of address
\address{142 Gates Building, Stanford University, CA 94305}
\address{ashesh@cs.cornell.edu, \url{www.cs.cornell.edu/~ashesh}}



\begin{resume} 
 

\section{Interests}
My research interest lies at the intersection of machine learning, robotics and computer vision. Broadly, I build machine learning systems \& algorithms for agents -- such as robots, cars etc. -- to learn from informative human signals at a large-scale. Most of my work has been in multi-modal sensor-rich robotic settings, for which I have developed sensory fusion deep learning architectures. I have developed and deployed algorithms on multiple robotic platforms (PR2, Baxter etc.), on cars, and crowd-sourcing systems. 
\iffalse

for humans to help machines -- such as robots, cars etc. -- become smart by implicitly providing
learning signals. I am particularly interested in learning from signals that are easy to
elicit at large-scale but are inherently weak and noisy. I am recently
exploiting the potential of crowd-sourcing feedback to teach robots good trajectories in
context-rich environments and using vision based human feedback for
assistive cars. 
\fi
\section{Education} 
 
{\bf Cornell University}, New York, USA  \hfill (2012-2016 (Exptd))
 
 \vspace{-4mm}
  \hspace {4mm } Ph.D. student, Computer Science 
 \begin{itemize} \itemsep -2pt  % reduce space between items
 \item PhD committee: Ashutosh Saxena (advisor), Thorsten Joachims, Doug James,
Robert Kleinberg, and Bart Selman.
 \item Learning from large-scale human signals for robots and assistive cars
  \item First author papers in NIPS, ICCV, SIGKDD, ICRA, ISRR, and IJRR
 \end{itemize}
 

 {\bf Indian Institute of Technology Delhi}, India  \hfill (2007-2012)
   
 \vspace{-4mm}
  \hspace {4mm } B.Tech. Electrical Engineering \& M.Tech. Information
and Communication Technology
 \begin{itemize} \itemsep -2pt  % reduce space between items
 \item Advisor: Prof. Manik Varma and Prof. S.V.N. Vishwanathan
 \item Thesis: Large-scale Algorithms for Multiple Kernel Learning
 \end{itemize}

\section{Experience}

{\bf Stanford University}, California, USA  \hfill (2014-Present)
 
 \vspace{-4mm}
  \hspace {4mm } Visiting Ph.D. student, Computer Science 
 \begin{itemize} \itemsep -2pt  % reduce space between items
 \item Working with Prof. Ashutosh Saxena and Prof. Silvio Savarese
 \item Leading Brain4Cars and Machine Learning lead on RoboBrain.
 \end{itemize}

{\bf Purdue University,} Indiana, USA \hfill (Summer 2011)
    
 \vspace{-3mm}
  \hspace {4mm }Research Intern, Statistics Department with Prof. S. V. N. Vishwanathan
 \begin{itemize} \itemsep -2pt  % reduce space between items
 \item Developed Multiple Kernel Learning algorithms to scale to Millions of
kernels
%(MKL) Support Vector Machine
% \item Scaling MKL to Millions of kernels on UCI data sets
 \end{itemize}

{\bf The Royal Bank of Scotland,} India  \hfill  (Summer 2010)
   
 %\vspace{-3mm}
 %\hspace {4mm }Software Engineering Intern

\begin{itemize} \itemsep -2pt 
\item Developed software to automate testing of web application GUI
%\item Deployed the software as an integral part of company’s software testing routine
\end{itemize}


 \section{Software}
 
 \textbf{NeuralModels}: A deep learning framework for quick prototyping of structures of Recurrent Neural Networks, Sensory-fusion architectures, and deep learning on graph structured data. Built on top of Theano.   (\href{https://github.com/asheshjain399/NeuralModels}{Github URL})
 
 \textbf{Brain4Cars}: A sensory-fusion architecture written in ROS for fusing
 multiple camera streams, tactile sensors, and GPS information. The package has
 integrated learning algorithms based on Bayesian networks for driver intention
 understanding and feature extraction modules.
 (\href{https://github.com/asheshjain399/ICCV2015_Brain4Cars}{URL: Data set and
 Bayesian network code}) (\href{https://github.com/asheshjain399/RNNexp}{URL: Deep
 learning architecture})

\textbf{SPG-GMKL}: Multiple kernel learning (MKL) with a million kernels. This is a generalized MKL tool kit based on spectral projected gradient. It can be used for optimizing arbitrary combination of kernels and regularizer. (\href{http://www.cs.cornell.edu/~ashesh/pubs/code/SPG-GMKL/download.html}{Download page})  
 \section{Publications\\ (in Chronological order)} 

  		A. Jain, H. S. Koppula, S. Soh, B. Raghavan, A. Singh, and A. Saxena.
Brain4Cars: Car That Knows Before You Do via Sensory-Fusion Deep Learning Architecture\\In IJRR, 2016 (In review) 

 		A. Jain, A. R. Zamir, S. Savarese, and A. Saxena.
Structural-RNN: Deep Learning on Spatio-Temporal Graphs\\In CVPR, 2016 (In review) 

		A. Jain, A. Singh, H. S. Koppua, S. Soh, and A. Saxena.
Recurrent Neural Networks for Driver Activity Anticipation via Sensory-Fusion Architecture.\\ In ICRA, 2016 (In review)
 
		A. Jain, H. S. Koppua, B. Raghavan, S. Soh, and A. Saxena.
Car That Knows Before You Do: Anticipating Maneuvers via Learning Temporal Driving Models.\\In ICCV, 2015 


		A. Jain and A. Saxena. Brain4Cars: Sensory-Fusion Recurrent Neural Models for Driver Activity Anticipation\\
In BayLearn Symposium, 2015 (\textbf{Full Oral}) 
 
  		A. Saxena, A. Jain, O. Sener, A. Jami, D. K. Misra and H. S. Koppula.
RoboBrain: Large-Scale Knowledge Engine for Robots.\\In ISRR, 2015.

  		A. Jain, S. Sharma, T. Joachims and  A. Saxena. Learning Preferences for
Manipulation Tasks from Online Coactive Feedback.\\In IJRR, 2015.
 
 		A. Jain, D. Das, J. K. Gupta and  A. Saxena. PlanIt: A Crowdsourcing
Approach for Learning to Plan Paths from Large Scale Preference Feedback.\\In ICRA, 2015
 
 		H. S. Koppula, A. Jain and A. Saxena. Anticipatory Planning for
Human-Robot Teams.\\In ISER, 2014.
 
		 A. Jain, B. Wojcik, T. Joachims and A. Saxena. Learning Trajectory Preferences for Manipulators via Iterative Improvement.\\In NIPS, 2013.

 		A. Jain, S. Sharma  and A. Saxena. Beyond Geometric Path Planning:
Learning Context-Driven Trajectory Preferences via Sub-optimal Feedback.\\In ISRR, 2013.

  		A. Jain, S. V. N. Vishwanathan and and M. Varma. SPG-GMKL:
Generalized multiple kernel learning with a million kernels.\\In  SIGKDD, 2012.

 \section{Research in  \\ Popular Press \\ } 
 \vspace{-4mm}
\hspace{-4mm}
\begin{itemize}
\item Brain4Cars appeared on \href{http://money.cnn.com/video/news/2015/04/16/brain4cars-can-predict-driver-error-cornell-stanford.cnnmoney/}{CNN technology} front page and \href{http://news.discovery.com/autos/future-of-transportation/car-predicts-driving-mistakes-before-they-happen-150417.htm}{Discovery News}. (2015)
\item Interviewed by \href{http://www.technologyreview.com/news/541866/this-car-knows-your-next-misstep-before-you-make-it/}{MIT Technology Review} for Car that knows before the driver makes a mistake. (2015)
\item RoboBrain covered by \href{http://www.nytimes.com/2014/09/02/science/robot-touch.html?_r=0}{The New York Times}, \href{http://www.wired.com/2014/08/robobrain/}{Wired}, and \href{http://www.technologyreview.com/view/533471/robobrain-the-worlds-first-knowledge-engine-for-robots/}{MIT Technology Review}. (2014)
\item Interviewed by \href{http://www.bbc.com/news/technology-25465672}{BBC World News} prime time show \textit{Click} for building robots
for supermarkets. Broadcasted to millions of viewers. (2014)
\item Research video on Teaching Robots from Human Signals received more than
\href{https://www.youtube.com/watch?v=uLktpkd7ojA}{100,000 hits} on YouTube in less than 24 hours. Research covered by Discovery Channel, FOX
News, IEEE Spectrum, Techcrunch and many others. (2013)
\item One of two students selected from IIT Delhi to present at the Indo-German
Winter Academy on semiconductor device physics and fabrication. (2010)
\item Director's Merit Award IIT Delhi, for highest semester grade point (2010)
\item Summer Undergraduate Research Award, IIT Delhi, for embedded systems
project on: Design and development of low cost optical densitometer for rural
paramedics (2009)
\end{itemize} 

\section{Projects 2012-Present}
{\bf Brain4Cars: Deep Learning and Perception for Smart Car Cabins} \hfill (2014 - Present)
    
 \vspace{-3mm}
  \hspace {4mm } Project lead: {\url{http://www.brain4cars.com}}
 \begin{itemize} \itemsep -2pt  % reduce space between items
\item Integrated car cabin with an array of cabin sensors (cameras, tactile sensors, wearable
devices, etc.) in order to extract valuable statistics about the driver.
\item Developed sensory-fusion deep learning architecture for fusing information from multiple sensors in order to predict the driver's future maneuvers.
 \end{itemize}
 
{\bf RoboBrain: Massive Knowledge Graph for Robots} \hfill (2014 - Present)
    
 \vspace{-3mm}
  \hspace {4mm } Leading PhD student: {\url{http://robobrain.me}}
 \begin{itemize} \itemsep -2pt  % reduce space between items
 \item Developed machine learning algorithms for knowledge graph combining
multi-modal data from Natural Language, Vision, 3D-perception and Robot
trajectories
\item Conceptualized \& built the distributed system architecture for managing multi-modal
data, learning from crowd-sourcing feedback and visualizing the knowledge graph. 
 \item Leading the project on major fronts: complete software stack,
knowledge graph building, crowd-sourcing feedback and probabilistic beliefs on
graph
%\item Building large-scale learning system involving: Amazon Elastic Cloud, Django, MongoDB and
%Graph Databases such as Neo4j
%\item Served more than 50,000 public hits in last two months
 \end{itemize}




{\bf PlanIt: A Crowd-sourcing Approach for Learning to Plan Paths} 
    
 %\vspace{-3mm}
 %\hspace {4mm } 
 \begin{itemize} \itemsep -2pt  % reduce space between items
 \item Built a machine learning system to teach robots good trajectories from
crowd-sourcing feedback. Project webpage: {\url{http://planit.cs.cornell.edu}} 
 \item Modeled user feedback as a generative process with latent user intentions  
 \end{itemize}

{\bf Beyond Geometric path planning: Path planning in context}

 \begin{itemize} \itemsep -2pt  % reduce space between items
 \item Developed an algorithm to learn user preferences over robot trajectories 
from sub-optimal coactive feedback 
 \item Deployed the algorithm on PR2 and Baxter robotic platforms and trained
them for various household-chores and grocery checkout tasks
 \item Received substantial attention from popular press such as FOX News, IEEE
Spectrum, Techcrunch, Discovery Channel and many others
 \end{itemize}

{\bf SPG-GMKL: Multiple kernel learning (MKL) with Millions of kernels}

\begin{itemize}
\item Developed MKL algorithm that scaled to Millions of kernels on UCI data
sets
\item The released software package is being used by many research groups
\end{itemize}



\iffalse  
\section{Teaching Experience}
 {\bf Cornell University,} Ithaca, NY, USA \hfill (2010-present)
    
 \vspace{-3mm}
  \hspace {4mm}  Teaching Assistant, Computer Science Department
 \begin{itemize} \itemsep -2pt  % reduce space between items
 \item CS 4740: Natural Language Processing, Spring 2013
 \item CS 4700: Artificial Intelligence, Fall 2012
 \end{itemize}
 
 {\bf Indian Institute of Technology Delhi}, India \hfill (2011 - 2012)
    
 \vspace{-3mm}
  \hspace {4mm}  Teaching Assistant, Electrical Engineering Department
 \begin{itemize} \itemsep -2pt  % reduce space between items
 \item EEL102: Introduction to Circuits, Spring 2011. Awarded with Undergraduate
 Teaching Assistant award.
 \item EEL101: Introduction to Circuits, Fall 2011
 \item EEL102: Introduction to Circuits, Spring 2012
 \end{itemize}
\fi

\section{Other \\ Activities}
	Workshop Organizer for Learning Plans with Context from Human Signals, RSS 2014\\
	
	\vspace{-7mm}
	Reviewer for NIPS, ICML, RSS, ICRA, AURO \\

\section{Research talks}


Keynote at the ICCV workshop on Autonomous driving. Title: Deep Learning for Spatio-Temporal Problems: On Cars, Humans, and Robots, Dec 2015

Invited talk at Zoox Labs, Dec 2015

Talk at University of Washington Seattle Department of Computer Science, Nov 2015

Invited talk at Qualcomm Deep Learning Research Center, Nov 2015

Oral at BayLearn 2015 on Brain4Cars: Sensory-fusion Recurrent Neural Networks 

Invited Talk at RSS Workshop on Model Learning for Human-Robot Communication, July 2015

Invited Talk at ICML Workshop on Machine Learning for Interactive Systems, July 2015

Invited Talk at ICRA Tutorial on Planning, Control, and Sensing for Safe Human-Robot Interaction, May 2015

Invited Talk at IIT Kanpur Department of Computer Science. RoboBrain and Learning from Weak Signals, Feb 2015

Stanford Semantics and Geometry Seminar. RoboBrain and Learning from Weak Signals, Feb 2015

Stanford Robotics Seminar. Learning from Weak Signals, Nov 2014

Introductory talk at LPCHS workshop RSS 2014. Learning from Humans. 

Cornell AI Seminar and ISRR 2013. Beyond Geometric Path Planning. 

ICML Robot Learning workshop 2013.

Oral at SIGKDD 2012. 

Invited spotlight at Mysore Park Workshop on Machine Learning 2012. 

Lecture at Indo-German Winter Acadmey 2010.
 
 
\section{Mentoring}
Led more than 20 Master and Undergraduate students from Cornell and Stanford University. Building the teams for Brain4Cars and RoboBrain. Below are few outstanding students.

\vspace{-4mm}
{\bf Vaibhav Aggarwal} (\textbf{won Cornell ELI 2014 award} for his work on PlanIt)

\vspace{-4mm}
{\bf Debarghya Das}, 2014 - Full year (now at Facebook)

\vspace{-4mm}
{\bf Jayesh Gupta}, Summer 2014 Intern from IIT Kanpur, India (now PhD Stanford)

 \vspace{-4mm}
{\bf Shikhar Sharma}, Summer 2013 Intern from IIT Kanpur, India (now PhD U of
Toronto)

 \vspace{-4mm}
{\bf Brian Wojcik}, Spring 2013 (now at Microsoft) 

 \vspace{-4mm}
{\bf Bharad Raghavan}, 2014 - Present, MS at Stanford

 \vspace{-4mm}
{\bf Shane Soh}, 2014 - Present, MS at Stanford

 \vspace{-4mm}
{\bf Avi Singh}, 2015 - Present, Summer Intern from IIT Kanpur

 \vspace{-4mm}
{\bf Siddhant Manocha}, 2015 - Present, Summer Intern from IIT Kanpur

 \vspace{-4mm}
{\bf Arpit Agarwal}, 2015 - Present, Summer Intern from IIT Kanpur

\iffalse
% Tabulate Computer Skills; p{3in} defines paragraph 3 inches wide
\section{Computer \\ Skills}
   \begin{tabular}{l p{3in}}
    \underline{Languages:} & C, C++, Java, Perl, Python, Matlab \\

     \underline{Grid computing:} & Hadoop (Pig and streaming Perl) \\   
      \underline{Operating Systems:} & Linux, OSX, Windows \\ 
 \end{tabular}

	\fi


\end{resume} 
\end{document} 



